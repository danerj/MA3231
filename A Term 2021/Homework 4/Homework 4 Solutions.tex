\documentclass{article}

\usepackage{amsmath}
\usepackage{amssymb}
\usepackage{amsfonts}
\usepackage{fullpage}
\usepackage{graphicx}
\usepackage{float}
\usepackage[shortlabels]{enumitem}

\setcounter{MaxMatrixCols}{20} 


\title{MA 3231 Homework 4}
\date{}
\author{}

\setlength\parindent{0pt}
\begin{document}
\maketitle

\begin{enumerate}

%%%% 1 %%%%
\item

\begin{enumerate}
\item 

Denote by $x_{ij}$ the number of cars shipped from vertex $i$ to vertex $j$. The linear program describing this problem is

$\min z = c^Tx$\\
subject to \\
$Ax = -b$\\
$x \leq r$ \\
$x \geq 0$ \\

$$
A = \begin{bmatrix}
-1 &1 &-1& -1& 1& 0& 0& 0& 0& 0 & 0 & 0 &0 & 0& 0 \\
1 &-1& 0& 0 &0 &1& 0 &0& 0 &0 &0 &1 &-1 &0 &0 \\
0 &0 &0 &1 &-1& 0& -1& 0& 0& 0& 0& 0& 0& 0 &0 \\
0 &0 &0&0 &0 &0 &1 &-1 &1& 0 &0 &0 &0 &0& 0 \\
0& 0& 1 &0 &0& 0& 0& 1& -1& -1& 1 &-1&1& 0 &0 \\
0 &0 &0 &0 &0 &-1& 0& 0 &0 &1& -1& 0& 0& -1& 1 \\
0& 0& 0& 0 &0 &0& 0 &0& 0& 0& 0& 0 &0 &1& -1
\end{bmatrix}
$$
$$
x = 
\begin{bmatrix}
x_{12} \\
x_{21} \\
x_{15} \\
x_{13} \\
x_{31} \\
x_{62} \\
x_{34} \\
x_{45} \\
x_{54} \\
x_{56} \\
x_{65} \\
x_{25} \\
x_{52} \\
x_{67} \\
x_{76}
\end{bmatrix},
c = 
\begin{bmatrix}
3.6\\
3.6\\
1\\
6\\
6\\
2.6\\
1\\
6\\
6\\
0\\
0\\
0\\
0\\
10.6\\
10.6\\
\end{bmatrix},
r = 
\begin{bmatrix}
70\\
70\\
3\\
30\\
30\\
5\\
2\\
20\\
20\\
20\\
20\\
30\\
30\\
20\\
20
\end{bmatrix},
b= 
\begin{bmatrix}
0 \\
8 \\
8 \\
-10 \\
0 \\
0 \\
-6
\end{bmatrix}
$$

\item

We do not need to add constraints that say that the variables must be integer-valued. According to the integral flow theorem, the optimal solution $x$ must be integer-valued since each edge has an integer-valued capacity and the supply and demand values are all integer-valued. 

While you can only ship an integer-valued number of cars on the railroad it does not follow that the value of the objective function at the optimal solution of the linear program in part a must also be integer-valued since it is the minimum cost of shipping the railroad cars. Note that some of the shipping costs are not integer-valued either. \\

\item 

The minimum cost is $\$163.40$ with:

$$
x = \begin{bmatrix}
x_{12} \\
x_{21} \\
x_{15} \\
x_{13} \\
x_{31} \\
x_{62} \\
x_{34} \\
x_{45} \\
x_{54} \\
x_{56} \\
x_{65} \\
x_{25} \\
x_{52} \\
x_{67} \\
x_{76}
\end{bmatrix}
= 
\begin{bmatrix}
3 \\
0 \\
3 \\
0\\
6 \\
0 \\
2 \\
0 \\
8 \\
20 \\
14 \\
30 \\
19 \\
6 \\
0
\end{bmatrix}
$$

\end{enumerate} 

\newpage
%%% 2 %%%
\item

\begin{enumerate}
\item
$\min z = c^Tx$\\
subject to \\
$A_1x = -b_1$\\
$A_2x = -b_2$\\
$x \leq r$ \\
$x \geq 0$ \\

where the vectors $x,c,$ and $r$ be are as in problem 1 and

$$
A_1 = \begin{bmatrix}
-1 &1& -1& -1 &1 &0 &0& 0 &0& 0 &0& 0& 0 &0& 0 \\
0& 0 &0&0& 0& 0& 1& -1& 1 &0 &0 &0& 0 &0 &0 \\
0 &0 &1 &0 &0 &0 &0 &1 &-1 &-1& 1 &-1& 1 &0 &0 \\
0 &0 &0 &0 &0 &-1& 0 &0&0 &1& -1& 0 &0 &-1&1 \\
0& 0 &0 &0 &0& 0 &0 &0 &0 &0& 0 &0 &0& 1&-1
\end{bmatrix}
$$
$$
A_2 = \begin{bmatrix}
1 &-1 &0& 0& 0& 1& 0 &0 &0& 0 &0 &1&-1 &0 &0 \\
0 &0 &0& 1& -1 &0 &-1 &0& 0& 0 &0 &0& 0 &0 &0 \\
\end{bmatrix}
$$
$$
b_1 = \begin{bmatrix}
0 \\
-10 \\
0 \\
0 \\
-6
\end{bmatrix},
b_2 = 
\begin{bmatrix}
20 \\
20
\end{bmatrix}
$$


\item The minimum cost is $\$113.60$ with:

$$
x = \begin{bmatrix}
x_{12} \\
x_{21} \\
x_{15} \\
x_{13} \\
x_{31} \\
x_{62} \\
x_{34} \\
x_{45} \\
x_{54} \\
x_{56} \\
x_{65} \\
x_{25} \\
x_{52} \\
x_{67} \\
x_{76}
\end{bmatrix}
= 
\begin{bmatrix}
0 \\
0 \\
0 \\
0 \\
0 \\
0 \\
2 \\
0 \\
8 \\
20 \\
8 \\
30 \\
10 \\
6 \\
0
\end{bmatrix}
$$
\end{enumerate}

\newpage
%%% 3 %%%
\item

Consider the smaller of the two networks. 

\begin{enumerate}
\item
Set up a max-flow linear program:

$-\min -x_{51}$ \\
subject to \\
$Ax = 0$\\
$x \leq r$ \\
$x \geq 0$ \\

$$
x = 
\begin{bmatrix}
x_{12} \\
x_{21} \\
x_{13} \\
x_{31} \\
x_{23} \\
x_{32} \\
x_{25} \\
x_{52} \\
x_{34} \\
x_{43} \\
x_{45} \\
x_{54} \\
x_{51} 
\end{bmatrix},
r = 
\begin{bmatrix}
1 \\
1\\
1\\
1\\
1\\
1\\
1\\
1\\
1\\
1\\
2\\
2\\
10000
\end{bmatrix}
$$
$$
A = 
\begin{bmatrix}
-1 &1 &-1& 1& 0 &0 &0& 0& 0 &0 &0& 0& 1 \\
1 &-1 &0 &0& -1 &1 &-1& 1 &0 &0 &0& 0 &0 \\
0 &0 &1& -1 &1 &-1 &0& 0 &-1& 1 &0 &0 &0\\
0 &0& 0& 0 &0 &0 &0 &0 &1& -1& -1 &1& 0\\
0 &0 &0& 0 &0 &0& 1& -1 &0 &0 &1& -1& -1
\end{bmatrix}
$$

The value of the objective function at the optimal solution to this program is equal to the value of the objective function for the optimal solution to the  dual problem, which is the min-cut linear program, by the strong duality theorem.

\item 

The optimal solution to the max-flow problem has objective function value 2. That means the optimal solution to the min-cut problem also has objective function value 2. Therefore the minimum number of lines that need to be cut is 2. 

\item

We could cut the line connecting vertices 1 and 2 and cut the line connecting vertices 1 and 3. 
Alternatively, we could cut the line connecting vertices 2 and 5 and cut the line connecting vertices 3 and 4. In either case we only need to cut 2 lines, which agrees with the previous result. 




\end{enumerate}

\end{enumerate}

\end{document}